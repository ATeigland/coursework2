
\documentclass[12pt,a4paper,titlepage]{article}
\usepackage[ansinew]{inputenc}
\usepackage{amsmath,amsfonts,amssymb,setspace,subfig,multirow,textcomp,booktabs,tabularx}
\usepackage{graphicx}
\usepackage{array}
\usepackage{multirow}
\usepackage[left=2.00cm, right=2.00cm]{geometry}
\usepackage{caption}
\usepackage[section,above,below]{placeins}
\usepackage{eurosym}
\usepackage[natbib=true,citestyle=science,bibstyle=numeric,sorting=none,maxnames=4]{biblatex}
\usepackage[flushleft]{threeparttable}
\usepackage{tikz,pst-plot}

\bibliography{gyroscopebib}
\title{Microfabricated 2 degree of freedom gyroscope - a case study}
\date{\today}
\author{Anders Teigland\\ MPhil in Micro and Nanotechnology Enterprise \\ Dept of Material Science and Metallurgy\\ Trinity Hall, University of Cambridge\\ email \texttt{at615@cam.ac.uk}\\ Module: NE 02 MEMS Design \\ Supervisor: Prof. Ashwin Seshia}

\begin{document}
\begin{spacing}{1.1}
\maketitle
\section{Introduction}
Gyroscopes are considered important devices in many household products. Amongst them are both the automotive industry, where gyroscopes are applied to improve both comfort and safety, and hand held devices like smart phones, tablets and in devices with image stabilization.

Several designs of the gyroscope has been proposed, but all pose either reliability issues or very complex processing is required compared to designs utilising some sort of vibratory sensing. The vibratory design relies on the Coriolis force acting on a vibrating mass that is, most commonly, optically or capacitively sensed.

In this paper we will optimise a design for a 2 degree of freedom (DOF) MEMS gyroscope. The design of such a device is shown in figure *. In the first part we will derive expressions for some of the parameters we will consider and apply some design constraints to these to arrive at a final design for the device.

\section{Derivations of device operation}
As we see from figure *, the device has beams acting as springs in both the drive and sense directions. We will consider these as cantilevers with a load applied to the end. Doing so yields the spring constant for a single beam expressed in equation \ref{eq:ksinglebeam} for a single beam. Expanding this into equation \ref{eq:kyorx} for the drive or sense directions, we arrive at an expression for the spring constant of the total system in both drive and sense directions by assuming that the 4 beams acting in each direction are identical. In these expressions E is the Youngs modulus and w, h and l the width, height and length of the beams indicated on figure *.

\begin{equation}
k_{single beam} = \dfrac{1}{4} \dfrac{E h w^3}{l^3}
\label{eq:ksinglebeam}
\end{equation}
\begin{equation}
k_{i,total} = 4 k_{single beam} = \dfrac{E h w_i^3}{l_i^3}
\label{eq:kyorx}
\end{equation}

With the proof mass oscillating in the drive direction one can describe the device as a 2-DOF vibratory system. The resulting equations of motions showed in equations \ref{eq:motiondrivedirection} and \ref{eq:motionsensedirection}. Here, $\omega_i$ is the resonance frequency in the i direction, $Q_i = \dfrac{m_i \omega_i}{b_i}$ is the quality factor, $b_i$ the damping factor, $\Omega$ the external rotation frequency and $F_d$ the electrostatic driving force. d and s subscripts denote sense (\textbf{x}) and drive (\textbf{y}) directions, respectively.

\begin{equation}
\ddot{y}+\dfrac{\omega_d}{Q_d} \dot{y} + \omega_d^2 y = \dfrac{F_d}{m} \sin(\omega_d t)
\label{eq:motiondrivedirection}
\end{equation}
\begin{equation}
\ddot{x} + \dfrac{\omega_s}{Q_s} \dot{x} + \omega^2 x + 2 \Omega \dot{y} = 0
\label{eq:motionsensedirection}
\end{equation}

Since gyroscopes generally are underdamped, it is reasonable to assume that the drive direction amplitude can be modelled as below.

\begin{equation}
y(t) = A_0 \sin(\omega_d t) = Q_d \dfrac{F_d}{k_y} \sin{\omega_d t}
\label{eq:yamplitudeast}
\end{equation}

By solving the system of differential equations\footnote{These expressions were found using Mathematica, and are identical to the once presented in %citation needed.
} 
above, we get the following expressions for the displacement and also a phase relative to the drive frequency.

\begin{equation}
x(t) = \dfrac{2 \Omega \cdot A_0 \omega_d}{\omega_s^2 \sqrt{\left(1-\left(\dfrac{\omega_d}{\omega_s}\right)^2\right)^2 + \left(\dfrac{\omega_d}{Q_s\cdot \omega_s}\right)}} \cdot \sin(\omega_d t - \varphi)
\label{eq:xoftraw}
\end{equation}
\begin{equation}
\varphi = \arctan \left(\dfrac{1}{Q_s \left(\dfrac{\omega_s}{\omega_d}- \dfrac{\omega_d}{\omega_s}\right)}\right)
\label{eq:phase}
\end{equation}

\begin{equation}
C(\omega) = \dfrac{1}{ \sqrt{\left(1-\left(\dfrac{\omega_d}{\omega_s}\right)^2\right)^2 + \left(\dfrac{\omega_d}{Q_s\cdot \omega_s}\right)}}
\label{eq:constant}
\end{equation}

\begin{equation}
\lvert X \rvert = \dfrac{2 \cdot \Omega \cdot A_0 \cdot C (\omega)}{\omega_s^2}
\label{eq:absolutedisplacementsense}
\end{equation}

In equation \ref{eq:bandwith} we see the bandwidth of a gyroscope at resonance, and by inserting into equation  \ref{eq:absolutedisplacementsense} we can analyse the bandwidth-sensitivity trade-off for the device. As seen in equation \ref{eq:bandwithsensitivitytradeoff}, the sensitivity of the device is heavily reliant on the bandwidth, in fact there is an inverse square dependence. This means that a narrow bandwidth is required. As the device is operated at ambient conditions, we will have to accept a loss in sensitivity as it is very difficult to achieve narrow bandwidths here.

\begin{equation}
B.W.= \dfrac{\omega_i}{Q_i}
\label{eq:bandwith}
\end{equation}
\begin{equation}
\dfrac{X}{\Omega}=\dfrac{2 \cdot A_0 \cdot C (\omega)}{\omega_s^2}=\dfrac{2 \cdot A_0 \cdot C (\omega)}{B.W.^2 \cdot Q_s^2}
\label{eq:bandwithsensitivitytradeoff}
\end{equation}

In addition, we need to consider the noise in the system. In this case, thermal noise will be present, caused by the Brownian force which can be expressed as in equation \ref{eq:BrownianForce}. By equating this with the Coriolis force, equation \ref{eq:Coriolisforce}, acting on the vibrating mass, we get an expression for the noise in the system, shown in equation \ref{eq:browniannoise}.

\begin{equation}
\lvert F_{Brownian}\rvert = \sqrt{4k_b T b_s B.W.} = \sqrt{\dfrac{4k_b T \omega_s m B.W.}{Q_s}}
\label{eq:BrownianForce}
\end{equation}

\begin{equation}
\lvert F_{Coriolis}\rvert = 2 m \lvert \dot{y} \rvert \Omega = 2 m \omega_d A_0 \Omega
\label{eq:Coriolisforce}
\end{equation}
\begin{equation}
2 m \omega_d A_0 \Omega = \sqrt{\dfrac{4k_b T \omega_s m B.W.}{Q_s}}
\label{eq:brownequcori}
\end{equation}
\begin{equation}
\dfrac{\Omega}{\sqrt{B.W.}} = \sqrt{\dfrac{k_b T \omega_s}{m Q_s A_0^2 \omega_d^2}}
\label{eq:browniannoise}
\end{equation}

Our last consideration is that of the driving force. In equations \ref{eq:combdrivecapacitance} and \ref{eq:capstoredenergy} we see the capacitance, $C$, and stored energy, $ U $, of a comb drive actuator, respectively. By differentiating the energy in the direction of movement of the actuator we get the expression for the force in equation \ref{eq:capforce}. In the following expressions, $\epsilon$ is the dielectric constant of the medium where the gyroscope operates, $l_0$ is the overlap length of the comb fingers, $h$ is the height of the fingers and, hence, the device, and $g_0$ is the nominal gap between the comb fingers and electrode as indicated on figure *.

\begin{equation}
C = \epsilon \dfrac{A_{overlap}}{g_0} = \epsilon \dfrac{2n \cdot (l_0 +y) h  }{g_0}
\label{eq:combdrivecapacitance}
\end{equation}
\begin{equation}
U = \dfrac{1}{2} C V^2
\label{eq:capstoredenergy}
\end{equation}
\begin{equation}
F = -\nabla U = - \dfrac{1}{2} \dfrac{\delta C}{\delta y} V^2 = -\dfrac{\epsilon \cdot n  \cdot h}{g_0} V^2
\label{eq:capforce}
\end{equation}

By applying different voltages, $V_1 = V_0 + v(t)$ and $V_2 = V_0 - v(t)$, to the top and bottom electrodes one induces an effective force acting on the mass\supercite{Chang12}. Here $V_0$ is a DC bias and $v(t)$ an applied AC bias.

\begin{equation}
F_{drive} = F_{top} - F_{top} = -4\dfrac{\epsilon \cdot n  \cdot h}{g_0} V_o v(t)
\end{equation} 



\section{Numerical Analysis}
To approach a functioning device we will now apply some of the design constraints to the derived expressions above to try to optimise device operation.In table \ref{tab:designconstraints}*, we see a summary of the design constraints that will be used below.

\begin{table}[htbp]
\caption{Table showing some of the design constraints}
\centering
\begin{tabular}{l|r}
\toprule
Maximum design area (mm) & {2 x 2} \\ 
Min dimensions of proof mass ($\mu$m) & {500 x 500} \\ 
Min beam width ($\mu$m) & 2 \\ 
Max beam deflection (\%) & 2 \\
Thickness of structure ($\mu$m) & 6 \\ 
Max flexure length($\mu$m) & 500 \\ 
Min electrode gap ($\mu$m) & 0.5 \\ 
Vertical gap between structure and substrate ($\mu$m) & 2 \\ \bottomrule
\end{tabular}
\label{tab:designconstraints}
\end{table}

\subsection{Flexure design}
From equation \ref{eq:browniannoise}, it is evident that the Brownian noise in the system is highly dependent on the amplitude of the driving vibrations. As the beam deflection is limited to 2 \% of the beam length it is therefore reasonable to chose the longest possible beams of 500 $\mu$m. Furthermore, we see from equation \ref{eq:yamplitudeast} that the amplitude is inversely proportional to the mechanical spring constant in the drive direction, we therefore wish to minimise this quantity by choosing the minimum beam width of 2 $\mu$m with the thickness of the structure fixed at 6 $\mu$m and Youngs modulus of silicon of 130 GPa assuming that the bending is in the [100]-direction \supercite{Bhushan97}.

\begin{equation}
k_d = \dfrac{E h w_d^3}{l_d^3}= 0.05 Nm^{-1}
\label{eq:drivespringconstant}
\end{equation}

Equation \ref{eq:drivespringconstant} along with the given resonance frequency in the drive direction makes it possible to calculate the proof mass as in equation \ref{eq:springconsttomass}.

\begin{equation}
\omega_d = \sqrt{\dfrac{k_d}{m}} \rightarrow m = \dfrac{k_d}{\omega_d^2} = 1.26 \cdot 10^{-11} kg
\label{eq:springconsttomass}
\end{equation}

If we assume a quadratic, seen from the top, proof mass we can find the dimensions as seen in  equation \ref{eq:flexdimensionsfirst} below. Here, $\rho_Si$ is the density of silicon, $V$ the volume of the mass and $l_m$ the mass dimensions in the xy-plane of the device.

\begin{equation}
\rho_Si = \dfrac{m}{V} = \dfrac{m}{l_m^2 h} \leftrightarrow l_m = \sqrt{\dfrac{m}{\rho_Si h}} = 30 \mu m
\label{eq:flexdimensionsfirst}
\end{equation}

As we see this result is far below the minimum mass dimensions and we will have to iterate our results to find suitable flexure dimensions. Furthermore, we will chose a rectangular proof mass with width 510 $\mu$m and the length maximised so that to mount the most electrodes possible on it. In table *, values for the 1st iteration are put.

\begin{table}[htbp]
\caption{Table showing iterated dimensions}
\centering
\begin{tabular}{lllllll}
\toprule
$w_d [\mu m]$ & $l_d [\mu m]$ & $k_d$[N/m] & m [$\mu g$] & $w_mass [\mu g]$ & $l_mass [\mu m]$ & $A_0 [\mu m]$ \\ \midrule
\multicolumn{1}{r}{5} & \multicolumn{1}{r}{130} & \multicolumn{1}{r}{44.38} & \multicolumn{1}{r}{11.2} & \multicolumn{1}{r}{510} & \multicolumn{1}{r}{1530} & \multicolumn{1}{r}{2.6} \\ \bottomrule
\end{tabular}
\label{tab:1stiteration}
\end{table}


\end{spacing}
\printbibliography
\end{document}
